\documentclass[a4paper,12pt]{article}
% decent example of doing mathematics and proofs in LaTeX.
% An Incredible degree of information can be found at
% http://en.wikibooks.org/wiki/LaTeX/Mathematics

% Use wide margins, but not quite so wide as fullpage.sty
\marginparwidth 0.1in 
\oddsidemargin 0.05in 
\evensidemargin 0.05in 
\marginparsep 0.05in
\topmargin 0.05in 
\textwidth 6in \textheight 8 in
% That's about enough definitions

\usepackage{amsmath, amsthm, amssymb, amsfonts}
\usepackage{mathtools}
\usepackage[utf8]{inputenc}
\usepackage[english]{babel}

\makeatletter
\renewenvironment{proof}[1][\proofname] {\par\pushQED{\qed}\normalfont\topsep6\p@\@plus6\p@\relax\trivlist\item[\hskip\labelsep\bfseries#1\@addpunct{.}]\ignorespaces}{\popQED\endtrivlist\@endpefalse}
\makeatother

\newtheoremstyle{break}
  {\topsep}{\topsep}%
  {\itshape}{}%
  {\bfseries}{}%
  {\newline}{}%
\theoremstyle{break}
\newtheorem{theorem}{Theorem}[section]
\newtheorem{corollary}{Corollary}[theorem]
\newtheorem{lemma}[theorem]{Lemma}
\newtheorem{definition}[theorem]{Definition}
\DeclarePairedDelimiter{\norm}{\lVert}{\rVert}


\begin{document}

\title{Calculus of Multivariate Functions}
\author{Hyeonsu Lyu, Department of Electrical Engineering, UNIST}
\date{July 3, 2018}
\maketitle

\section{Inverse and Implicit function theorem}
\begin{theorem} [The inverse function theorem] 
    \label{thm:InverseFunctionTheorem}
    Suppose $f$ is a $\mathcal{C}^1-mapping$ of an open set $E\subset \mathbb{R}^n$ into $R^n$. $f'(a)$ is invertible for some $a\in E$, and $b=f(a)$. Then,\\
    \indent (a)  there exists open sets $U$ and $V$ in $\mathbb{R}^n$ such that $a\in U, b\in V$, $f$ is one-to-one on $U$, and $f(U)=V$ \\
    \indent (b)  if $g$ is the inverse of $f$, [which exists, by (a)], defined in $V$ by
    \begin{align}
        g(f(x))=x\ \  (x\in U),
    \end{align}
    then $g\in \mathcal{C}(V).$
\end{theorem}

\begin{theorem} [The implicit function theorem]
    \label{thm:ImplicitFunctionTheorem}
    Let $f$ be a $\mathcal{C}^1-mapping$ of an open set $E\subset \mathbb{R}^{n+m}$ into $\mathbb{R}^n$, such that $f(a,b)=0$ for some point $(a,b)\in E$.\\
    \indent Put $A=f'(a,b)$ and assuem that $A_{\textit{x}}$ is invertible.\\
    \indent Then there exists open sets $U\subset \mathbb{R}^{n+m}$ and $b \in W$, habing the following property:\\
    \indent To every $y\in W$ corresponds a unique $x$ such that\\
    \begin{align}
        (x,y)\in\ U\ \ \ \ and \ \ \ f(x,y)=0.
    \end{align}
    \indent if this $x$ is defined to be $g(y)$, then $g$ is a $\mathcal{C}^1-mapping$ of $W$ into $\mathbb{R}^{n}$, $g(b)=a$,
    \begin{align}
        f(g(y),y)=0\ \ (y\in W),
    \end{align}
    \indent and
    \begin{align}
        g'(b)=-(A_x)^{-1}A_y.
    \end{align}
    \indent The function $g$ is “implicitly” defined by (4). Hence the name of the theorem.
\end{theorem}

\begin{proof} [Proof of theorem \ref{thm:InverseFunctionTheorem}]
    \indent (a) Put $f'(a)=A$, and choose $\lambda$ so that
    \begin{align}
        2\lambda\norm{A^{-1}}=1.
        \label{p1:1}
    \end{align}
    \indent since $f'$ is continuous at $a$, there is an open ball $U\subset E$, with center at $a$, such that
    \begin{align}
        \norm{f'(x)-A}<\lambda\ \ (x\in U).
        \label{p1:2}
    \end{align}
    We associate to each $y\in \mathbb{R}^n$ a function $\varphi$, defined by
    \begin{align}
        \varphi(x)=x+A^{-1}(y-f(x)) \ \ (x \in E)
        \label{p1:3}
    \end{align}
    Note that $f(x)=y$ if and only if x is a fixed point of $\varphi$.
    Since $\varphi'(x)=I-A^{-1}f'(x)=A^{-1}(A-f'(x))$, (\ref{p1:1}) and (\ref{p1:2}) imply that
    \begin{align}
        \norm{\varphi'(x)}<\frac{1}{2}\ \ (x\in U)
        \label{p1:4}
    \end{align}
    Hence
    \begin{align}
        |\varphi(x_1)-\varphi(x_2)|\leq \frac{1}{2}|x_1-x_2|\ \ \ \ (x_1,x_2 \in U),
        \label{p1:5}
    \end{align}
    Then, by contraction theorem, it follows that $\varphi$ has at most one fixed point in U, so that $f(x)=y$ for at most one $x\in U$.
    Thus $f$ is $1-1$ in $U$.

    Next, put $V=f(U)$, and pick $y_0 \in V$. Then $y_0=f(x_0)$ for some $x_0\in U$. Let $B$ be an open ball with center at $x_0$ and radius $r>0$, so small that its closure $\bar{B}$ lies in U. We will show that $y\in V$ whenever $|y-y_0|<\lambda r$. This proves that V is open. (This proves that every $y\in V$ is interior point.) \\

    Fix $y$, $|y-y_0| < \lambda r$. with $\varphi$ as in (\ref{p1:3}),
    \begin{align*}
        \varphi(x)-x_0|=|A^{-1}(y-y_0)|<\norm{A^{-1}}\lambda r=\frac{r}{2}.
    \end{align*}
    if $x \in \bar{B}$, it therefore follows from (\ref{p1:5}) that
    \begin{align*}
        \varphi(x)-x_0| &\leq |\varphi(x) - \varphi(x_0)|+|\varphi(x_0)-x_0| \\
        &<\frac{1}{2}|x-x_0|+\frac{r}{2}\leq r;
    \end{align*}
    hence $\varphi(x) \in B$. Note that (\ref{p1:5}) holds if $x_1,x2\in\bar{B}$.

    Thus $\varphi$ is a contraction of $\bar{B}$ into $\bar{B}$. Being a closed subset of $\mathbb{R}^n$, $\bar{B}$ is complete. By contraction theorem, $\varphi$ has a fixed point $x\in \bar{B}$. For this $x, f(x)=y$. Thus, $y\in f(\bar{B})\subset f(U)=V$.

    This proves part (a) of the theorem.
    
    (b) Pick $y\in V$, $y+k\in V$. Then there exists $x\in U$, $x+h \in U$, so that $y=f(x),y+k=f(x+h)$. With $\varphi$ as in (\ref{p1:3}),
    \begin{align}
        \varphi{x+h}-\varphi{x}=h+A^{-1}[f(x)-f(x+h)h-A^{-1}k.
        \label{p1:6}
    \end{align}
    by (\ref{p1:5}), $|h-A^{-1}k|\leq\frac{1}{2}|h|$. Hence $|A^{-1}k|\geq\frac{1}{2}|h|,$ and
    \begin{align}
        |h|\leq2\norm{A^{-1}}|k|=\lambda^{-1}|k|
        \label{p1:7}
    \end{align}

    By (\ref{p1:1}), (\ref{p1:2}) and theorem 9.8[Rudin], $f'(x) has an inverse, say T.$ Since 
    \begin{align*}
        g(y+k)-g(y)-Tk=h-Tk=-T[f(x+h)-f(x)-f'(x)h],
    \end{align*}
    (\ref{p1:7}) implies
    \begin{align*}
        \frac{|g(y+k)-g(y)-Tk}{|k|}\leq\frac{\norm{T}}{\lambda} \cdot\frac{|f(x+h)-f(x)-f'(x)h|}{|h|}.
    \end{align*}
    
    As $k\to0$, (\ref{p1:7}) shows that $h\to0$. The right side of the last inequality thus tends to 0. Hence the same is true of the left. We have thus proved that $g'(y)=T$. But $T$ was chosen to be the inverse of $f'(x)=f'(g(y))$. Thus
    \begin{align}
        g'(y)={f'(g(y))}^{-1}\ \ \ (y\in V).
        \label{p1:8}
    \end{align}
     
    Finally, note that $g$ is a continuous mapping of V onto U (Since g is differentiable), that $f'$ is a continuous mapping of $U$ into the set $\Omega$ of all invertible elements of $L(\mathbb{R}^n$, and that inversion is a continuous mapping of $\Omega$ onto $\Omega$, by theorem 9.8[Rudin]. If we combine these facts with (\ref{p1:8}), we see that $g\in\mathcal{c}'(V)$.

\end{proof}



\end{document}
