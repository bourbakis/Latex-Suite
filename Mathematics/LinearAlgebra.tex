\documentclass[a4paper,12pt]{article}
% decent example of doing mathematics and proofs in LaTeX.
% An Incredible degree of information can be found at
% http://en.wikibooks.org/wiki/LaTeX/Mathematics

% Use wide margins, but not quite so wide as fullpage.sty
\marginparwidth 0.1in 
\oddsidemargin 0.05in 
\evensidemargin 0.05in 
\marginparsep 0.05in
\topmargin 0.05in 
\textwidth 6in \textheight 8 in
% That's about enough definitions

\usepackage{amsmath}
\usepackage{amsthm}
\usepackage{amssymb}
\usepackage[utf8]{inputenc}
\usepackage[english]{babel}

\makeatletter
\renewenvironment{proof}[1][\proofname] {\par\pushQED{\qed}\normalfont\topsep6\p@\@plus6\p@\relax\trivlist\item[\hskip\labelsep\bfseries#1\@addpunct{.}]\ignorespaces}{\popQED\endtrivlist\@endpefalse}
\makeatother

\newtheoremstyle{break}
  {\topsep}{\topsep}%
  {\itshape}{}%
  {\bfseries}{}%
  {\newline}{}%
\theoremstyle{break}
\newtheorem{theorem}{Theorem}[section]
\newtheorem{corollary}{Corollary}[theorem]
\newtheorem{lemma}[theorem]{Lemma}
\newtheorem{definition}[theorem]{Definition}


\begin{document}

\title{Linear Algebra Theorems}
\author{Hyeonsu Lyu, Department of Electrical Engineering, UNIST}
\date{June 29, 2018}
\maketitle

\section{Linear Transformation}
\begin{definition} [Linear Transformation from $\mathbb{R}^n$ to $\mathbb{R}^m$]
    A \textit{linear transformation} $T : \mathbb{R}^n\to\mathbb{R}^m$ is a mapping such that for all scarars $\textit{a}$ and all $\overrightarrow{v}, \overrightarrow{w} \in \mathbb{R}^n,$
\begin{align}
    T(\overrightarrow{v}+\overrightarrow{w}=T(\overrightarrow{v}+\overrightarrow{w})\ and\ T(a\overrightarrow{v})=aT(\overrightarrow{v})
\end{align}
\end{definition}

\begin{theorem} [Matrices and Linear transformations]\label{thm:mlt}
    1. Any $m\times n$ matrix A defines a linear transformation $T:\mathbb{R}^n\to\mathbb{R}^m$ by matrix multiplication:
    \begin{align}
        T(\overrightarrow{v})=A\overrightarrow{v}.
    \end{align}
        2. Every linear transformation $T:\mathbb{R}^n\to\mathbb{R}^m$ is given by multiplication by the $m\times n$ matrix $[T]$:
    \begin{align}
        T(\overrightarrow{v})=[T]\overrightarrow{v}
    \end{align}
        ,where the \textit{i}th column of $[T]$ is $T(\overrightarrow{e_i}).$
\end{theorem}

\begin{proof}[Proof of theorem \ref{thm:mlt}] \hfill \\
1. By definition of \textit{linear transformation}, mapping (2) is \textit{linear}.\\
2. For any vector $\overrightarrow{v}\in \mathbb{R}^n$, $v$ can be represented as basis of $\mathbb{R}^n$, $\{\overrightarrow{e_1},\overrightarrow{e_2}, \dots, \overrightarrow{e_n}\}$.
\begin{align}
    \overrightarrow{v} = v_1\overrightarrow{e_1}+v_2\overrightarrow{e_2}+\dots+v_n\overrightarrow{e_n}
\end{align}
Since the linearity of mapping T,
\begin{align*}
    T(\overrightarrow{v})&=v_1T(\overrightarrow{e_1})+v_2T(\overrightarrow{e_2})+\dots+v_nT(\overrightarrow{e_n})\\
    &=[T(\overrightarrow{e_1})|T(\overrightarrow{e_2})|\dots|T(\overrightarrow{e_n})]\begin{bmatrix}v_1 \\ v_2 \\ \vdots \\ v_n\end{bmatrix}\\
    &= [T]\overrightarrow{v}
\end{align*}
\end{proof}

\begin{definition} [Linear Transformation is onto]
    Let $A$ be $m\times n$ matrix. The followings are logically equivalent: \\
    \indent1. The function $Ax=b$ is onto $\mathbb{R}^m$.\\
    \indent2. For each $b\in \mathbb{R^m}$, the equation Ax=b has a solution.\\
    \indent3. $b\in \mathbb{R^m}$ is a linear combination of the columns of $A$.\\
    \indent4. The columns of $A$ span $\mathbb{R^m}$.\\
    \indent5. $A$ has a pivot position in every row.\\
    \indent6. $A$ has rank m\\
\end{definition}

\begin{definition} [Linear Transformation is 1-1]
    Let $A$ be an $m\times n$ matrix. The followings are logically equivalent:\\
    \indent1. The function $Ax=b$ is $1-1$.\\
    \indent2. The equation $Ax=b$ has at most on solution for every b.\\
    \indent3. The equation $Ax=0$ has only the trivial solution.\\
    \indent4. The columns of A are linearly independent.\\
    \indent5. A has a pivot posiition in every column.\\
    \indent6. A has rank n.\\
\end{definition}

Related theroems : Inverse function theorem, Implicit function theorem, Tangent space on implicit function.

\section{Another Section}

\end{document}

